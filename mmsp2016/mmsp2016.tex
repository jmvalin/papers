%% LyX 2.1.4 created this file.  For more info, see http://www.lyx.org/.
%% Do not edit unless you really know what you are doing.
\documentclass[english,conference,10pt]{IEEEtran}
\usepackage[T1]{fontenc}
\usepackage[latin9]{inputenc}
\usepackage{babel}
\begin{document}

\title{Daala: Building A Next-Generation Video Codec From Unconventional
Technology}


\author{\author{\IEEEauthorblockN{Some authors\IEEEauthorrefmark{1}\IEEEauthorrefmark{2}}
\IEEEauthorblockA{\IEEEauthorrefmark{1}Mozilla Corporation, Mountain View, USA}\\
\IEEEauthorblockA{\IEEEauthorrefmark{2}Xiph.Org Foundation, USA}
}}
\maketitle
\begin{abstract}
Daala is a new royalty-free video codec based on perceptually-driven
coding techniques.
\end{abstract}


\section{Introduction}

Daala is a royalty-free video codec designed to avoid traditional
patent-encumbered techniques used in most current video codecs. One
of the goals of the Daala codec is to explore techniques that are
very different from those typically used in most codecs.


\section{Daala Techniques}

One of the goals of the Daala codec is to explore techniques that
are very different from those typically used in most codecs. Most
of these techniques have been fundamental to Daala since the initial
stages of the project. 


\subsection{Lapping}

Daala uses lapped transforms~\cite{MalvarS89,Tran2003} rather than
a regular DCT followed by a deblocking filter. 

This reduces blocking artifacts but prevents the use of standard pixel-based
intra-prediction techniques~\cite{DaedeDCC}. Instead, we use a simple
frequency-domain intra predictor that only handles horizontal and
vertical directions~\cite{EggePCS}. 

Also, DC coefficients are combined recursively using a Haar transform,
up to the level of 64x64 superblocks. 


\subsection{Multi-Symbol Entropy Coder}

Most recent video codecs encode information using binary arithmetic
coding, meaning that each symbol can only take two values. The Daala
range coder supports up to 16 values per symbol, making it possible
to encode fewer symbols~\cite{derfTools}. This is equivalent to
coding up to four binary values in parallel and reduces serial dependencies. 

To achieve this, Daala uses a multiply-free range coder approximation~\cite{stuiver1998piecewise}. 


\subsection{Overlapped Block Motion Compensation}

Lapping -> Straight BMA is a no-go


\subsection{Perceptual Vector Quantization}

In the majority of video codecs, motion-compensated reference frame
is subtracted from the input frame to compute a residual, which is
then transformed, and coded using scalar quantization. Daala differs
significantly from that description. Not only does it use vector quantization
rather than scalar quantization, but -- most importantly -- the motion-compensated
reference is never subtracted from the input frame. Instead, both
the motion-compensated reference is used to build a transformation
that makes the input easier to code. This technique is called perceptual
vector quantization (PVQ)~\cite{valin2015spie}.

Perceptual vector quantization originates in the pyramid vector quantizer
previously used for music in the Opus audio codec~\cite{ValinAES}.

PVQ makes it possible to take into account masking effects with no
extra signaling. 


\subsection{Chroma from Luma (CfL) Prediction}

Because of its structure, PVQ makes it especially easy to predict
chroma planes from the luma plane. Daala's chroma from luma (CfL)~\cite{egge2015spie}
prediction uses the luma transform coefficients to predict the chroma
transform coefficients.


\subsection{Directional Deringing Filter}

Let $x\left(n\right)$ denote a 1-dimensional signal and $w_{k}$
denote filter tap weights, a linear finite impulse response (FIR)
filter with unit DC response is defined as
\begin{equation}
y\left(n\right)=\frac{1}{\sum_{k}w_{k}}\sum_{k}w_{k}x\left(n+k\right)\ ,\label{eq:FIR1}
\end{equation}
which can alternatively be written as
\begin{equation}
y\left(n\right)=x\left(n\right)+\frac{1}{\sum_{k}w_{k}}\sum_{k,k\neq0}w_{k}\left[x\left(n+k\right)-x\left(n\right)\right]\ .\label{eq:FIR2}
\end{equation}
The main advantage of expressing a filter in the form of Eq.~(\ref{eq:FIR2})
is that the normalization term $\frac{1}{\sum_{k}w_{k}}$ can be approximated
relatively coarsely without affecting the unit gain for DC. This makes
it easy to use small integers for the weights $w_{k}$.

The disadvantage of linear filters for removing ringing artifacts
is that they tend to also cause blurring. To reduce the amount of
blurring, the conditional replacement filter used in Daala excludes
the signal taps $x\left(n+k\right)$ that would cause blurring and
replaces them with $x\left(n\right)$ instead. This is determined
by whether $x\left(n+k\right)$ differs from $x\left(n\right)$ by
more than a threshold $T$. The FIR filter in Eq.~(\ref{eq:FIR2})
then becomes a conditional replacement filter expressed as
\begin{equation}
y\left(n\right)=x\left(n\right)+\frac{1}{\sum_{k}w_{k}}\sum_{k,k\neq0}w_{k}R\left(x\left(n+k\right)-x\left(n\right),T\right)\ ,\label{eq:CRF}
\end{equation}
where
\begin{equation}
R\left(x,T\right)=\left\{ \begin{array}{ll}
x & ,\ \left|x\right|<T\\
0 & ,\ \mathrm{otherwise}
\end{array}\right.\ .
\end{equation}


To further reduce the risk of blurring the decoded image, the conditional
replacement filter is applied along the main direction of the edges
in each 8x8 block. The algorithm for finding directions is described
in~\cite{ValinDeringing} and is the same as the one previously used
in the paint deringing filter. For each 8x8 block, it determines which
of eight different directions best represents the content of the block.
It can be efficiently implemented in SIMD. A 7-tap conditional replacement
filter is applied in Daala for a single pixel in a 8x8 block. The
process is repeated for each pixel in each block being filtered.

To reduce ringing in very smooth regions of the image, the filter
is applied a second time to combine multiple output values of the
first filter. The second filter is applied either vertically or horizontally
-- in the direction most orthogonal to the one used in the first filter.
For example, for a 45-degree direction, the second filter would be
applied vertically. The combined effect of the two filters is a separable
deringing filter that covers a total of 35~pixel taps.


\section{Alliance for Open Media AV1 codec}


\section{Results}


\section{Conclusion}

\bibliographystyle{IEEEtran}
\bibliography{daala}

\end{document}
